% Example LaTeX document for GP111 - note % sign indicates a comment
\documentstyle[11pt]{article}
% Default margins are too wide all the way around. I reset them here
\setlength{\topmargin}{-.5in}
\setlength{\textheight}{9in}
\setlength{\oddsidemargin}{.125in}
\setlength{\textwidth}{6.25in}
\begin{document}
\author{Cyril Allen\\
Harvard Extension School}
\renewcommand{\today}{who knows}



\section{Synthetic Workload Generation}

To verify that the work presented in this paper is improving upon the already
existing filesystem, it's necessary to generate a repeatable, synthetic
workload on the system in addition to simulations that verify the correct
behavior of the replica selection algorithm. To perform this workload
generation, I use fio[1].

\subsection{fio - flexible I/O tester}

fio is a tool that simulates I/O via spawning multiple threads or processes to
do a particular type of I/O action and it has the ability to generate latency
and bandwidth logs for each job [2]. Using fio removes the need to write
separate test programs for each type of workload and allows configuration via a
file that outlines the workload parameters. In addition to the convenience of
configuraiton files, fio provides a JSON output format that allows for easy
graphing of latency/bandwidth information.






Biblio:
1. Axboe, J. (2014). Fio. urlhttp://freecode. comlprojects/fio.
2. Axboe, J., & Carroll, A. fio (1)-Linux man page. URl: http://linux. die. net/man/1/fio.
